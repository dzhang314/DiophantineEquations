\documentclass[11pt]{article}

\title{Notes on Diophantine Equations}
\author{David K. Zhang}
\date{Last modified \today}

\usepackage{amsmath}
\usepackage{amssymb}
\usepackage[margin=1.25in]{geometry}
\usepackage{parskip}

\newcommand{\Z}{{\mathbb{Z}}}
\newcommand{\R}{{\mathbb{R}}}
\newcommand{\dfntxt}[1]{{\textbf{\textit{#1}}}}

\begin{document}
\maketitle

A \dfntxt{Diophantine equation} is a polynomial equation with integer coefficients, in any number of variables, for which we seek integer solutions. For example, the Diophantine equation
\[ x^2 + y^2 = z^2 \qquad\qquad x, y, z \in \Z \]
defines the set of Pythagorean triples $(x, y, z) \in \Z^3$. Every Diophantine equation can be written in the form
\[ P(x_1, \dots, x_n) = 0 \qquad\qquad x_1, \dots, x_n \in \Z \]
for some polynomial $P \in \Z[x_1, \dots, x_n]$. For example, the preceding Diophantine equation can be represented by the polynomial $P(x, y, z) = x^2 + y^2 - z^2$. In this article, we will freely identify Diophantine equations and polynomials with integer coefficients.

\begin{itemize}
	\item Any Diophantine equation of the form
	\[ x_1 + Q(x_2, \dots, x_n) = 0 \]
	can be trivially solved by assigning arbitrary integer values to $x_2, \dots, x_n$ and taking $x_1 = -Q(x_2, \dots, x_n)$.

	More generally, we say that a variable $x_1$ \dfntxt{occurs linearly} in a polynomial $P \in \Z[x_1, \dots, x_n]$ if $P$ can be written in the form
	\[ P(x_1, \dots, x_n) = a x_1 x_2^{j_2} \cdots x_k^{j_k} + Q(x_{k+1}, \dots, x_n) \]
	for some polynomial $Q \in \Z[x_{k+1}, \dots, x_n]$. The corresponding Diophantine equation
	\[ a x_1 x_2^{j_2} \cdots x_k^{j_k} + Q(x_{k+1}, \dots, x_n) = 0 \]
	can be solved by assigning $x_2 = \dots = x_k = 1$ and checking whether it is possible for $Q(x_{k+1}, \dots, x_n)$ to be a multiple of $a$, which can be tested in finite time by computing $Q(x_{k+1}, \dots, x_n)$ modulo $a$ for all values of $x_{k+1}, \dots, x_n \in \{0, \dots, a - 1\}$. This observation allows us to exclude all Diophantine equations that contain a linear variable.

	\item If a polynomial $P \in \Z[x_1, \dots, x_n]$ has no roots in $\R^n$, then it obviously cannot have any roots in $\Z^n$. This condition can be checked in finite time\footnote{The computational complexity of the cylindrical algebraic decomposition algorithm is doubly-exponential in the number of variables in the polynomial under consideration.} using the algorithms of real algebraic geometry, such as cylindrical algebraic decomposition.

	More generally, if the set of roots of $P$ in $\R^n$ is bounded, then there are only finitely many points of $\Z^n$ that we need to check in order to determine whether $P$ has an integer-valued root. The boundedness of this set can be checked by performing quantifier elimination on the following formula:
	\[ \forall t \in \R\ \exists x_1, \dots, x_n \in \R: P(x_1, \dots, x_n) = 0 \wedge x_1^2 + \dots + x_n^2 \ge t \]
	This observation allows us to exclude all Diophantine equations whose set of real-valued solutions is empty or bounded.
\end{itemize}

\textbf{Lemma:} Let $n \in \Z$. If $p$ is an odd prime factor of $n^2 + 1$, then $p \equiv 1 \pmod 4$.

\textit{Proof:} Let $p$ be an odd prime. If $p \mid (n^2 + 1)$, then $n^2 \equiv -1 \pmod p$, which shows that $-1$ is a quadratic residue modulo $p$. Using the properties of the Legendre symbol, it follows that
\[ 1 = \left( \frac{-1}{p} \right) = (-1)^{(p-1)/2} \]
which shows that $(p-1)/2$ is an even number. \textbf{QED}

\textbf{Theorem:} The Diophantine equation
\[ x^2 y + y^2 + z^2 + 1 = 0 \]
has no integer-valued solutions.

\textit{Proof:} We proceed by contradiction. Using the substitution $y \mapsto -y$, we can rewrite this equation as
\[ y(x^2 - y) = z^2 + 1. \]
In other words, this equation expresses $z^2 + 1$ as a product of two integers whose sum is a perfect square. Observe that both of the factors, $y$ and $x^2 - y$, must be positive.

Recall that every perfect square is congruent to 0 or 1 modulo 4. Hence, the right-hand side of this equation is congruent to 1 or 2 modulo 4. There are two ways to express each of these quantities as a product modulo 4:
\[ 1 \times 1 \equiv 3 \times 3 \equiv 1 \pmod 4 \qquad\qquad 1 \times 2 \equiv 3 \times 2 \equiv 2 \pmod 4 \]
Thus, the two factors on the left-hand side must be congruent to $(1, 1)$, $(3, 3)$, $(1, 2)$, or $(2, 3)$ modulo 4. The first three options are impossible because their sum modulo 4 is not 0 or 1. The preceding lemma implies that the final option is impossible, since no odd prime factor of $z^2 + 1$ (and hence, by induction, no odd factor of $z^2 + 1$) can be congruent to 3 modulo 4. \textbf{QED}

\textbf{Theorem:} The Diophantine equation
\[ x^2 y + 3y + z^2 + 1 = 0 \]
has no integer-valued solutions.

\textit{Proof:} Substitute $y \mapsto -y$ to obtain the equation $y(x^2 + 3) = z^2 + 1$. Working modulo 4, the right-hand side is either 1 or 2, while $x^2 + 3$ is either 0 or 3. Observe that $x^2 + 3 \equiv 0$ is impossible because it would force the right-hand side to be 0, and $x^2 + 3 \equiv 3$ is impossible by the preceding lemma. \textbf{QED}

\end{document}